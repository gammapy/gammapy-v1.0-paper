\section{Introduction}
\label{sec:introduction}



Very high energy cosmic gamma-ray reaching Earth interact in the atmosphere and
create a large shower of secondary particles that can be observed from the ground.
Ground-based gamma-ray astronomy relies on this phenomenon to detect the
primary gamma-ray photons and measure their direction and energy.
It covers the energy range from fews tens of GeV up to the PeV.
Two main instrument categories exist \citep{2015CRPhy..16..610D}.
Imaging Atmospheric Cherenkov Telescopes (IACT) make images of atmospheric showers
by detecting the Cherenkov radiation emitted by the cascading charged particles and
use these images to reconstruct the properties of the incident particle.

Particle samplers detect directly particles from the tail of the shower when it reaches
the ground. These instruments have very large field-of-view, large duty-cycle but higher
energy threshold and usually have lower signal to noise ratios compared to IACTs.

Ground based gamma-ray astronomy has been historically structured
by experiments run by independent collaborations relying
on their own proprietary data and analysis software developed as part of the
instrument. While this model has been very successful so far, it does not
permit easy combination of data from several instruments and is therefore
a limitation for interoperability of existing facilities.

Yet, the data-reduction workflow of different IACTs of the current generation is
remarkably similar. After data calibration, shower events are reconstructed and
gamma/hadron separation is applied to build lists of gamma-ray like events.
The latter are then used to derive scientific results, such as spectra, sky maps
or light curves, taking into account the Instrument Response Functions (IRF).
The information in this high data level is independent on
the data reduction, and eventually of the detection technique. This implies,
for example, that data from IACT and WCD can be represented within the same
model. The efforts to prototype a format usable by various instruments
converged in the so-called \textit{Data Format for Gamma-ray Astronomy}
initiative~\citep{gadf_proc, gadf_universe}, abbreviated in
\texttt{gamma-astro-data-format} (GADF). The latter proposes prototypical
specifications to produce files based on the flexible image transport system
(FITS) format~\citep{fits} encapsulating this high-level information. This is
realized by storing a list of gamma-like events with their measured quantites
(energy, direction, arrival time) and a parametrisation of the response of the
system. (see Sec.~\ref{ssec:gammapy-data} and Sec.~\ref{ssec:gammapy-irf} for
more information).

The Cherenkov Telescope Array (CTA) will be the first instrument to be operated
as an open observatory in the domain. Its high level data will be shared publicly after
some proprietary period and the software required to analyze it will be distributed
as well.


Python has become extremely popular as a scientific programming language
in particular in the field of data sciences. The success is
mostly attributed to the simple and easy to learn syntax, the ability to act as
a "glue" language between different programming languages, the rich eco-system
of packages and the open and supportive community.

 one of the
standard programming  languages for astronomy \footnote{Citation missing} as
well as data sciences in  general \footnote{Citation missing}.

Astronomical data analysis software written in Python existed since 2000. e.g.,
sherpa~\citep{sherpa-2011, sherpa-2009}, or for gamma-ray even PyFACT
~\citep{pyfact}.

The short-term success of Python lead to a prolifaration of packages, until
\astropy~\citep{astropy} was created in 2012. Astropy is and Gammapy is a
Python package for gamma-ray astronomy.

% TODO: describe Context

% TODO: describe goals
The H.E.S.S. collaboration released a limited test dataset (about 50 hours of
observations taken between 2004 and 2008) based  on the GADF DL3 format \citep{HESS_DR1}.
This data release served as a basis for analysis tools validation \cite[see e.g.]{Mohrmann2019}.
% TODO : discuss joint crab paper?

TODO: Figure 1: Data -> Gammapy -> Spectra etc with some details

Basic idea: build on Numpy and Astropy, use Python stack

TODO: Figure 2: Gammapy software stack


\begin{figure}[t]
	\centering
	\includegraphics[height=0.5\textwidth,
		angle=270]{static/gammapy-big-picture} \caption{ Gammapy is a Python package
		for high-level gamma-ray data analysis. Using event lists, exposures and point
		spread functions as input you can use it to generate science results such as
		images, spectra, light curves or source catalogs. So far it has been used to
		simulate and analyse H.E.S.S., CTA and \textit{Fermi}-LAT data, hopefully it
		will also be applied to e.g., VERITAS, MAGIC or HAWC data in the future. }
	\label{fig:big-picture}

\end{figure}


