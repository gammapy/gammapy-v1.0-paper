\section{Analysis Workflow Overview}
\label{sec:analysis-workflow-overview}
\todo{Regis and Fabio F.}

\section{Gammapy package}
\label{sec:gammapy-package}

The \gammapy package is structured into multiple sub-packages which mostly
follow the stages in the data reduction workflow.

\subsection{Overview}
\label{ssec:overview}
\begin{figure*}[t]
	\centering
	\includegraphics[width=1.\textwidth]{figures/data_flow.pdf}
	\caption{
		Gammapy sub-package structure and data analysis workflow. }
	\label{fig:data_flow} \end{figure*}

Outline: * List typical analysis use cases * Can use from Python and Jupyter ->
show Figure with Jupyter notebook here. * Gammapy code structure * How Numpy
and Astropy is used

Figures: * Add a Figure showing dataflow in a typical application DL3 at the
top, spectrum, map, lightcurve, fit results at the bottom. Mention major
classes in between (DataStore, EventList, Map, MapMaker, MapFit, …) * Probably
not: Figure showing sub-packages and how they relate (gammapy.data and
gammapy.irf at the base, then gammapy.maps, etc. * The code example Figure how
to make a counts map, to explain how the package works.

\todo{How to sort the sub-packages? After data flow or alphabetically? What
	about maps?} \subsection{gammapy.data}
\label{ssec:gammapy-data}
\todo{Cosimo Nigro}
The \verb|gammapy.data| sub-package provides access to DL3 level data and
observation handling.

\begin{figure}

	\import{code-examples/generated/}{gp_data}

	\caption{Using gammapy.data to access DL3 level data with a DataStore}
	\label{fig*:minted:gp_data}
\end{figure}
\subsection{gammapy.irf}
\label{ssec:gammapy-irf}
%
%\begin{figure}
%	%	\import{code-examples/generated/}{gp_makers}
%
%	\caption{In Fig.~\ref{ig*:minted:irf_examples}, we show some example of Aeff,
%		PSF, Edisp and BKG read and plotted from a typical \irf.}
%	\label{ig*:minted:irf_examples}
%\end{figure}
%
The \code{gammapy.irf} sub-package contains all classes and functionality
to handle IRFs in a variety of formats.
Usually, \irfs store instrument properties in the form of multi-dimensional
tables, with quantities expressed in terms of energy (true or reconstructed),
off-axis angles or cartesian detector coordinates. The main information stored in
the common \gammaray \irfs are the effective area (Aeff), energy dispersion
(Edisp), point spread function (PSF) and background rate (BKG). The \code{gammapy.irf}
sub-package can open and access specific \irf extensions,
interpolate and evaluate the quantities of interest on both energy and spatial
axes, convert their format or units in different kinds, plot or write them into
output files. In the following, we list the main classes of the
sub-package:

\paragraph{Effective area:}
\gammapy provides the class \code{EffectiveAreaTable2D} to
manage Aeff, which is usually defined in terms of true energy and offset angle.
The class functionalities offer the possibility to read from files or to create
it from scratch. {\code{EffectiveAreaTable2D} can also convert, interpolate,
write, and evaluate the effective area for a given energy and offset angles, or
even plot the multi-dimensional Aeff table.


\paragraph{Point spread function:}
\gammapy allows user to treat different kinds of PSFs,
in particular, parametric multi-dimensional gaussians (\code{EnergyDependentMultiGaussPSF})
or King profile functions (\code{PSFKing})
one. The \code{EnergyDependentMultiGaussPSF} class is able to handle up to three
gaussians, defined in terms of amplitudes and sigma given for each true energy
and offset angle bin. Similarly, \code{PSFKing takes} into account the gamma and
sigma parameters as defined here. The \code{ParametricPSF} allows to create a
PSF with a representation different from gaussian(s) or King profile(s).
Finally, the user can take advantage from the \code{PSFMap} class, which creates
a multi-dimensional map of the PSF in WCS coordinates. At each position, a PSF
kernel map (\code{PSFKernel}) provides a PSF as a function of the true energy.
The creation of PSF kernel maps, where the PSF is defined for each
sky-position, is also given to the user. The latter two can speed up analyses.

\paragraph{Energy dispersion:}
Edisp, in \iact, is generally given in terms of the so
called migration parameter ($\mu$), which is defined as the ratio between the
reconstructed energy and the true energy. This ratio should be as close as one
and its dispersion can assume the shape of a gaussian (or even more complex
distributions). Migration parameter is given at each offset angle and
reconstructed energy. The main sub-classes are \code{EnergyDispersion2D},
designed to interpret Edisp, \code{EDispKernelMap}, which builds an Edisp kernel
map, i.e., a 4-dimensional WCS map where at each sky-position is associated an
Edisp kernel. The latter is a representation of the Edisp as a function of the
true energy only thanks to the sub-class \code{EDispKernel}.

\paragraph{Background model:}
The BKG can be represented in \gammapy as either 1) a 2D map (\code{Background2D})
of count rate normalised per steradians and energy at different
energies and offset-angles or 2) as rate per steradians and energy, as a
function of reconstructed energy and detector coordinates (\code{Background3D}).
In the former, the background is expected to follow a radially symmetric shape,
while in the latter, it can be more complex.

 \subsection{gammapy.maps}
\label{ssec:gammapy-maps}
The \code{gammapy.maps} sub-package provides classes that represent data
structures associated with a set of coordinates or a region on a sphere. In
addition it allows to handle an arbitrary number of non-spatial data
dimensions, such as time or energy. It is organized around three types of
structures: geometries, sky-maps and map axes, which inherit from the base
classes \code{Geom}, \code{Map} and \code{MapAxis} respectively.

The geometry object defines the pixelization scheme and map boundaries. It also
provides methods to transform between sky and pixel coordinates. Maps consist
of a geometry instance together with a Numpy data array containing the corresponding
map values. Map axes contain a sequence of ordered values which define bins on
a given dimension, spatial or not. Map axes can have physical units attached to
them, as well as define non-linear spaced bins. All map classes support operations such
as arithmetic operations with unit support, up- and downsampling along
extra axes, interpolation, resampling of extra axes, interactive visualisation
in notebooks and interpolation onto different geometries.

The sub-package provides a uniform API for the FITS World Coordinate System
(WCS), the HEALPix pixelization and region-based data structure
(see Figure~\ref{ig*:minted:gp_maps}).

\begin{figure}
	\import{code-examples/generated/}{gp_maps}

	\caption{
        Using \code{gammapy.maps} to create a WCS, a HEALPix and a region
		based data structures. The initialisation parameters include
        consistently the positions of the center of the map, the pixel
        size, the extend of the map as well as the energy axis definition.
        The energy minimum and maximum values for the creation of the
        \code{MapAxis} object can be defined as strings also specifying the
        unit. Region definitions can be passed as strings following
        the DS9 region specifications \url{http://ds9.si.edu/doc/ref/region.html}.
        }
    \label{ig*:minted:gp_maps}
\end{figure}

% itemize because it helps me write, could also just be paragraphs...
\subsubsection{WCS Maps}
The FITS WCS pixelization supports a different number of projections to
represent celestial spherical coordinates in a regular rectangular grid.
Gammapy provides full support to data structures using this pixelization
scheme. For details see ~\cite{Calabretta2002}. This pixelisation
is typically used for smaller regions of interests, such as pointed
observations.


\subsubsection{HEALPix Maps}
This pixelization scheme ~\citep{Calabretta2002} provides a
subdivision of a sphere in which each pixel covers the same surface area as
every other pixel. As a consequence, however, pixel shapes are no longer
rectangular, or regular.
This pixelisation is typically used for all-sky data, such as data
from the \hawc or \fermi observatory. \gammapy natively supports
the multiscale definiton of the HEALPix pixelisation and thus
allows for easy up and downsampling of the data. In addition to
the all-sky map, \gammapy also supports a local HEALPix
pixelisation where the size of the map is constrained to a given
radius.
For local nighbourhood operations, such as convolution \gammapy relies
on projecting the HEALPix data to a local tangential WCS grid.

\subsubsection{Region Maps}
In this case, instead of a fine spatial grid
dividing a rectangular sky region, the spatial dimension is reduced to a single
bin with an arbitrary shape, describing a region in the sky with that same
shape. Typically they are is used together with a non-spatial dimension, for
example an energy axis, to represent how a quantity varies in that dimension
inside the corresponding region. The region is represented on the local
tangential projection see \cite{Astropy regions}.

Additionally, the \code{MapAxis} class provides a uniform API for axes representing
bins on any physical quantity, such as energy or angular offset. The special
case of time is covered by the dedicated \code{TimeMapAxis}, which allows time bins to
be non-contiguous, as it is often the case with observation time-stamps. The
generic class \code{LabelMapAxis} allows the creation of axes for non-numeric
entries.

\subsection{gammapy.makers}
\label{ssec:gammapy-makers}
%
\begin{figure}
    \import{code-examples/generated/}{gp_makers}
	\caption{
        Using \code{gammapy.makers} to reduce DL3 level data into a
		\code{MapDataset}. All \code{Maker} classes take
        the configuration on initialisation of the class.
    }
	\label{ig*:minted:gp_makers}
\end{figure}
%
The data reduction step includes all tasks required to process and prepare
\gammaray data from the DL3 level to modeling and fitting. The \code{gammapy.makers} sub-package
contains the various classes and functions required to do so. First, events are
binned and IRFs are interpolated and projected onto the chosen analysis
geometry. This produces counts, exposure, background, psf and energy dispersion
maps. The \code{MapDatasetMaker} and \code{SpectrumDatasetMaker} are
responsible for this task, see Fig~\ref{ig*:minted:gp_makers}.

Because the background models suffer from strong uncertainties it is required
to correct them from the data themselves. Several techniques are commonly used
in \gammaray astronomy such as field-of-view background normalization or
background measurement in reflected regions, see~\cite{Berge07}.
Specific \code{Makers} such as the \code{FoVBackgroundMaker} or the
\code{ReflectedRegionsBackgroundMaker} are in charge of this step.

Finally, to limit other sources of systematic uncertainties, a data validity
domain is determined by the \code{SafeMaskMaker}. It can be used to limit the
extent of the field of view used or to limit the energy range to e.g., a domain
where the energy reconstruction bias is below a given value.
%Laura: put it here just so that it shows up on the PDF...
\subsection{gammapy.datasets}
\label{ssec:gammapy-datasets}
\todo{Atreyee Sinha}

The end product of the data reduction process described in Section \ref{ssec:gammapy-makers} are a set of binned counts, background and IRF maps, at the DL4 level. The gammapy.datasets subpackage contains classes to bundle together binned data along with associated models and the likelihood, which provides an interface to the Fit class (Sec \ref{ssec:gammapy-modeling}) for modeling and fitting purposes. Depending upon the type of analysis and the associated statistics, different types of Datasets are supported. MapDataset is used for 3D (spectral and morphological) fitting, and a 1D spectral fitting is done using SpectrumDatastet. While the default statistics for both of these is Cash, their corresponding OnOff versions are adapted for the case where the background is measured from real off counts, and support wstat statistics. The predicted counts are computed by convolution of the models with the associate IRFs. Fitting of precomputed flux points is enabled through FluxPointsDatasets, using chi2 statistics. Multiple datasets of same or different types can be  bundled together in Datasets (e.g.: Figure \ref{fig*:minted:gp_datasets}), where the likelihood from each constituent member is added, thus facilitating joint fitting across different observations, and even different instruments across different wavelengths. Datasets also provide functionalities for manipulating reduced data, eg: stacking, sub-grouping, plotting, etc. Users can also create their customized datasets for implementing modified likelihood methods. 

\begin{figure}
	\import{code-examples/generated/}{gp_datasets}
	\caption{A Datasets container with FluxPointsDataset and MapDataset}
	\label{fig*:minted:gp_datasets}
\end{figure}


\subsection{gammapy.modeling}
\label{ssec:gammapy-modeling}
\todo{Quentin Remy}

gammapy.modeling contains all the functionality related to modeling and fitting
data. This includes spectral, spatial and temporal model classes, as well as
the fit and parameter API.

\subsection{Models}
\label{ssec:models}

The models are grouped into the following categories:

\begin{itemize}
	\item SpectralModel: models to describe spectral shapes of sources
	\item SpatialModel: models to describe spatial shapes (morphologies) of sources
	\item TemporalModel: models to describe temporal flux evolution of sources, such as
	      light and phase curves

\end{itemize}

The models follow a naming scheme which contains the category as a suffix to
the class name.

The  Spectral Models include a special class of Normed models, which have a
dimension-less normalisation. These spectral models feature a norm parameter
instead of amplitude and are named using the NormSpectralModel suffix. They
must be used along with another spectral model, as a multiplicative correction
factor according to their spectral shape. They can be typically used for
adjusting template based models, or adding a EBL correction to some analytic
model. The analytic Spatial models are all normalized such as they integrate to
unity over the sky but the template Spatial models may not, so in that special
case they have to be combined with a NormSpectralModel.

The SkyModel is a factorised model that combine the spectral, spatial and
temporal model components (by default the spatial and temporal components are
optional). SkyModel objects represents additive emission components, usually
sources or diffuse emission, although a single source can also be modeled by
multiple components. To handle list of multiple SkyModel components, Gammapy
has a Models class.

The model gallery provides a visual overview of the available models in
Gammapy. Most of the analytic models  commonly used in gamma-ray astronomy are
built-in. We also offer a wrapper to radiative models implemented in the Naima
package~\cite{naima}. The modeling framework can be easily extended with
user-defined models. For example agnpy models that describe leptonic radiative
processes in jetted Active Galactic Nuclei (AGN) can wrapped into
gammapy~\citep[see section3.5 of ][]{2021arXiv211214573N} .

\begin{figure}
	\import{code-examples/generated/}{gp_models}
	\caption{Using gammapy.modeling.models}
	\label{fig*:minted:gp_models}
\end{figure}

\subsection{Fit}
\label{ssec:fit}

The Fit class provides methods to fit, i.e. optimise parameters and estimate
parameter errors and correlations. It interfaces with a Datasets object, which
in turn is connected to a Models object containing the model parameters in its
Parameters object.  Models can be unique for a given dataset, or contribute to
multiple datasets and thus provide links, allowing e.g. to do a joint fit to
multiple IACT datasets, or to a joint IACT and \textit{Fermi}-LAT dataset. Many
examples are given in the tutorials.

The Fit class provides a uniform interface to multiple fitting backends:
“minuit”~\citep{iminuit}, “scipy”,~\citep{2020SciPy-NMeth}, and
“sherpa”~\citep{sherpa-2005,sherpa-2011}. Note that, for now, covariance matrix
and errors are computed only for the fitting with MINUIT. However depending on
the problem other optimizers can better perform, so sometimes it can be useful
to run a pre-fit with alternative optimization methods. In future we plan to
extend the supported Fit backend, including for example MCMC solutions.
\footnote{a prototype is available in gammapy-recipes,
	\url{https://gammapy.github.io/gammapy-recipes/_build/html/notebooks/mcmc-sampling-emcee/mcmc_sampling.html}
}

\subsection{gammapy.stats}
\label{ssec:gammapy-stats}
\todo{Regis Terrier}
Statistics methods

In Gammapy, the fit statistics are log-likelihood functions normalized like chi-squares,
i.e. they follow the expression $2 \times log L$, where $L$ is the likelihood function used.
To account for the event nature of gamma-ray data, the likelihood functions implemented
in Gammapy for Poisson distribution the Cash
\subsection{gammapy.estimators}
\label{ssec:gammapy-estimators}
\todo{Axel Donath}
The \verb|gammapy.estimators| sub-module features methods to compute
flux points, light curves, flux maps, flux profiles from data.

The initial fine binning of \verb|MapDataset| is grouped into larger bins.

Internal representation with a reference spectral model and an array of
normalisation values given in energy, time and spatial bins.

No unfolding.

\begin{itemize}
	\item Regrouping of dataset bins in time, energy etc.
	\item Reference spectral model scaled in energy bin
	\item "Forward folding" / "Backward folding": but there is no difference between the two, backwards folding is forward folding with one bin
	\item Likelihood profiles
	\item Uniform N-dimensional data structure
	\item Uniform API for plotting etc.
	\item \verb|FluxMaps| and \verb|FluxPoints|
	\item Serialisation to multiple formats, Astropy's \verb|Table| and \verb|BinnedTimeSeries|
	\item Additional quantities for debugging, such as predicted counts, fit convergence, sum of fit statistics
	\item
\end{itemize}

\begin{figure}
	\import{code-examples/generated/}{gp_estimators}
	\caption{Using the TSMapEstimator from gammapy.estimators to compute a sqrt(TS) map.}
	\label{fig*:minted:gp_estimators}
\end{figure}


\subsection{gammapy.analysis}
\label{ssec:gammapy-analysis}
\todo{Jose Enrique writes this...}
High level analysis API
\subsection{gammapy.visualisation}
\label{ssec:gammapy-visualisation}
\todo{Axel Donath}
Plotters etc.

\subsection{gammapy.astro}
\label{ssec:gammapy-astro}
\todo{Atreyee}

The gammapy.astro sub-package contains utility functions for studying physical scenarios in high energy astrophysics.
The dark matter module computes the so called J-factors and the gamma-ray spectra expected from annihilation of dark matter in different channels according to the recipe described in \cite{2011JCAP...03..051C}. In the source sub-module, dedicated classes exist for modeling galactic sources according to some come physical models, eg: SNR evolution models \citep{1950RSPSA.201..159T, 1999ApJS..120..299T}, evolution of PWN during the free expansion phase \citep{2006ARA&A..44...17G} or computation of physical parameters in a pulsar assumed to be a simple dipole. There are also dedicated tools for simulating synthetic populations based on physical models derived from observational or theoretical considerations for different classes of Galactic VHE gamma-ray emitters: PWNe, SNRs \cite{1998ApJ...504..761C}, pulsars \cite{2006ApJ...643..332F, 2006MNRAS.372..777L, 2004A&A...422..545Y} and gamma-ray binaries. While the present list of use cases is rather preliminary, this can be enriched with time with by users and/or developers according to future needs.
\subsection{gammapy.catalog}
\label{ssec:gammapy-catalog}
\todo{Atreyee}
Gamma-ray catalog access

Comprehensive source catalogs are increasingly being provided by many high
energy astrophysics experiments. gammapy.catalog provides a convenient access
to some common gamma-ray catalogs. A global catalog table is provided, along
with source models flux points and light curves (if available) for individual
objects, which are internally created from the supplied FITS tables. This
module works independently from the rest of the package, and the required
catalogs are supplied in GAMMAPY\_DATA. The overview of currenly supported
catalogs and the correponding \gammapy classes is shown in Table~\ref{tab:catalogs}.
Newly released relevant catalogs will be added in future.

\begin{table*}
    \begin{center}
        \begin{tabular}{llll}
         \hline
         Class & Shortcut & Description & Reference\\
         \hline
         \code{SourceCatalog3FGL} & \code{"3fgl"} & Third catalog of \fermi sources & \cite{3FGL} \\
         \code{SourceCatalog4FGL} & \code{"4fgl"} & Fourth catalog of \fermi  sources & \cite{4FGL} \\
         \code{SourceCatalog2FHL} & \code{"2fhl"} & Second catalog high energy \fermi  sources & \cite{2FHL} \\
         \code{SourceCatalog3FHL} & \code{"3fhl"} & Third catalog high energy \fermi  sources & \cite{3FHL} \\
         \code{SourceCatalog2HWC} & \code{"2hwc"} & Second catalog of \hawc sources & \cite{2HWC} \\
         \code{SourceCatalog3HWC} & \code{"3hwc"} & Third catalog of \hawc sources & \cite{3HWC} \\
         \code{SourceCatalogHGPS} & \code{"hgps"} & \hess Galactic Plane Survey catalog & \cite{HGPS} \\
         \code{SourceCatalogGammaCat} & \code{"gammacat"} & Open source data collection & \citep{gamma-cat} \\
         \hline
         \end{tabular}
    \end{center}
    \label{tab:catalogs}
    \caption{Built-in catalogs.}
\end{table*}

\begin{figure}
	\import{code-examples/generated/}{gp_catalogs}
	\caption{Using gammapy.catalogs: Accessing underlying model, flux points and
		lightcurve from the Fermi-LAT 4FGL catalog for the blazar PKS 2155-304}
	\label{codeexample:data}
\end{figure}

\subsection{gammapy.utils}
\label{ssec:gammapy-utils}
Utility functions...

