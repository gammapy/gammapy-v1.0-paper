\subsection{Surveys, catalogs, and population studies}

Sky surveys have a large potential for new source detections, and new phenomena discovery. They also offer less selection bias to perform source population studies over a large set of coherently detected and modelled objects.
Early versions of \textit{gammapy} where developed in parallel of the preparation of the HESS Galactic plane survey catalog \citep[HGPS][]{2018A&A...612A...1H} and the associated PWN and SNR populations studies \citep{2018A&A...612A...2H, 2018A&A...612A...3H}.

The CTA first Data challenge (DC1) proposed a large set of simulated observations covering the programs proposed in the Key science Projects for CTA  \citep{2019scta.book.....C}, including in particular extra-Galactic and Galactic surveys.
The studies performed on these simulations not only offered a first glimpse on what could be the sky seen by CTA, but also the opportunity to test the software, improve its performance, and identify the needs in term of parallelisation to process the large datasets provided by the surveys. Note that the CTA-DC1 simulations were performed with the \textit{ctools} package \citep{2016A&A...593A...1K} and analysed with both \textit{ctools} and \textit{gammapy} packages in order to cross-validate them. More recently new simulations of the CTA-GPS were analysed in the view of optimizing the production of the future CTA catalogs and associated populations studies \citep{2021arXiv210903729R}.
Future plans of the CTA observatory include the preparation of a new Data Challenge simulated with \textit{gammapy} that is meant to be released publicly in order to reach a broader community.
