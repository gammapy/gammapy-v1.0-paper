\subsection{gammapy.estimators}
\label{ssec:gammapy-estimators}
\todo{Axel Donath}
The \verb|gammapy.estimators| sub-module contains methods to compute flux
points, light curves, flux profiles, flux maps and  from data.

The light curve and flux profile estimation are based on the
flux points estimation.

The spectral energy distribution (SED)...

There are four common quantities used for the representation of
\gammaray spectra and flux points.

\begin{table*}
    \begin{center}
        \begin{tabular}{lll}
         \hline
         Type & Description & Unit Equivalency \\
         \hline
         dnde & Differential flux at a given energy & $\mathrm{TeV^{-1}~cm^{2}~s^{-1}}$ \\
         e2dnde & Differential flux at a given energy  & $\mathrm{TeV~cm^{2}~s^{-1}}$ \\
         flux & Integrated flux in agiven energy range & $\mathrm{cm^{2}~s^{-1}}$ \\
         eflux & Integrated energy flux in agiven energy range & $\mathrm{erg~cm^{2}~s^{-1}}$
        \end{tabular}
    \end{center}
    \label{tab:sed_types}
    \caption{Definition of sed types.}
\end{table*}



The main data structure for the result of is the `FluxMaps` class.
The internal representation is based on a reference spectral model
and an array of normalisation values given in energy, time and spatial bins,
the so called "norm".
The actual flux values are obtained by mutiplication of the norm with the
reference flux.
This models allows for a representation of flux values, wich are idependent
of the SED type.


The initial fine binning of \verb|MapDataset| is grouped into larger bins.



No unfolding.

\begin{itemize}
	\item Regrouping of dataset bins in time, energy etc.
	\item Reference spectral model scaled in energy bin
	\item "Forward folding" / "Backward folding": but there is no
	      difference between the two, backwards folding is forward folding with one bin
	\item Likelihood profiles \item Uniform N-dimensional data structure \item
	      Uniform API for plotting etc. \item \verb|FluxMaps| and \verb|FluxPoints| \item
	      Serialisation to multiple formats, Astropy's \verb|Table| and
	      \verb|BinnedTimeSeries|
	\item Additional quantities for debugging, such as predicted counts, fit convergence,
	      sum of fit statistics \item \end{itemize}

\begin{figure}
	\import{code-examples/generated/}{gp_estimators}
	\caption{Using the TSMapEstimator from gammapy.estimators to compute a
		sqrt(TS) map.} \label{fig*:minted:gp_estimators} \end{figure}
