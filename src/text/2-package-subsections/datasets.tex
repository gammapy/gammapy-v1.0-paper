\subsection{gammapy.datasets}
\label{ssec:gammapy-datasets}
\todo{Atreyee Sinha}

\begin{figure}
	\import{code-examples/generated/}{gp_datasets}
	\caption{A Datasets container with a \code{FluxPointsDataset} and \code{MapDataset}}
	\label{fig*:minted:gp_datasets}
\end{figure}

The end product of the data reduction process described in Section
\ref{ssec:gammapy-makers} are a set of binned counts, background and IRF maps,
at the DL4 level. The \code{gammapy.datasets} subpackage contains classes to bundle
together binned data along with associated models and the likelihood, which
provides an interface to the Fit class (Sec \ref{ssec:gammapy-modeling}) for
modeling and fitting purposes. Depending upon the type of analysis and the
associated statistics, different types of Datasets are supported. \code{MapDataset} is
used for 3D (spectral and morphological) fitting, and a 1D spectral fitting is
done using \code{SpectrumDataset}. While the default statistics for both of these is
Cash, their corresponding on off versions are adapted for the case where the
background is measured from real off counts, and support wstat statistics. The
predicted counts are computed by convolution of the models with the associated
IRFs. Fitting of precomputed flux points is enabled through \code{FluxPointsDataset},
using chi2 statistics. Multiple datasets of same or different types can be
bundled together in \code{Datasets} (e.g., Figure \ref{fig*:minted:gp_datasets}),
where the likelihood from each constituent member is added, thus facilitating
joint fitting across different observations, and even different instruments
across different wavelengths. Datasets also provide functionalities for
manipulating reduced data, eg: stacking, sub-grouping, plotting, etc. Users can
also create their customized datasets for implementing modified likelihood
methods.
