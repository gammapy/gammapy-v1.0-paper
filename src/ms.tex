%                                                                 aa.dem
% AA vers. 9.0, LaTeX class for Astronomy & Astrophysics
% demonstration file
%                                                       (c) EDP Sciences
%-----------------------------------------------------------------------
%
%\documentclass[referee]{aa} % for a referee version
%\documentclass[onecolumn]{aa} % for a paper on 1 column
%\documentclass[longauth]{aa} % for the long lists of affiliations
%\documentclass[rnote]{aa} % for the research notes
%\documentclass[letter]{aa} % for the letters
%\documentclass[bibyear]{aa} % if the references are not structured
%                              according to the author-year natbib style

% \documentclass[]{aa}
\documentclass[traditabstract, longauth]{aa}

\usepackage{graphicx}
\usepackage{url}
\usepackage{xcolor}
\usepackage{listings}
\usepackage{import}
\usepackage{fancyvrb}
\usepackage{color}
\usepackage[utf8]{inputenc}

\input{code-examples/minted.tex}
\newcommand{\todo}[1]{\textcolor{red}{TODO: #1}\PackageWarning{TODO:}{#1!}}

\newcommand{\code}[1]{\texttt{#1}}

\begin{document}

\newcommand{\PythonUrl}{\url{http://fits.gsfc.nasa.gov/}\xspace}
\newcommand{\FitsUrl}{\url{http://fits.gsfc.nasa.gov/}\xspace}
\newcommand{\GammapyUrl}{\url{http://gammapy.org}\xspace}
\newcommand{\GadfUrl}{\url{http://gamma-astro-data-formats.readthedocs.io/}\xspace}
\newcommand{\ReadthedocsUrl}{\url{https://readthedocs.org/}\xspace}
\newcommand{\TravisUrl}{\url{https://www.travis-ci.org/}\xspace}

\newcommand{\NaimaUrl}{\url{https://github.com/zblz/naima}\xspace}


% Note: not sure if we want to use that ... doesn't look too pretty
\newcommand{\astropy}{\texttt{Astropy}\xspace}
\newcommand{\gammapy}{\texttt{Gammapy}\xspace}

\newcommand{\hess}{H.E.S.S.~}
\newcommand{\hawc}{HAWC~}
\newcommand{\iact}{IACT~}
\newcommand{\iacts}{IACTs~}
\newcommand{\cta}{CTA~}
\newcommand{\swgo}{SWGO~}
\newcommand{\irf}{IRF~}
\newcommand{\irfs}{IRFs~}
\newcommand{\fermi}{Fermi-LAT~}
\newcommand{\gammaray}{$\gamma$-ray~}
\newcommand{\gammarays}{$\gamma$-rays~}
\newcommand{\gadf}{GADF~}



% Front matter
\title{Gammapy: A Python package for gamma-ray astronomy}
\titlerunning{Gammapy}
\authorrunning{Deil, Donath, Terrier et al.}

\author{
	Axel Donath \inst{\ref{inst:3}} \and
	Christoph Deil \inst{\ref{inst:8}} \and
	Régis Terrier \inst{\ref{inst:1}} \and
	Johannes King \inst{\ref{inst:15}} \and
	Jose Enrique Ruiz \inst{\ref{inst:11}} \and
	Quentin Remy \inst{\ref{inst:12}} \and
	Léa Jouvin \inst{\ref{inst:unknown}} \and
	Atreyee Sinha \inst{\ref{inst:7}} \and
	Matthew Wood \inst{\ref{inst:unknown}} \and
	Fabio Pintore \inst{\ref{inst:10}} \and
	Manuel Paz Arribas \inst{\ref{inst:unknown}} \and
	Laura Olivera \inst{\ref{inst:12}} \and
	Luca Giunti \inst{\ref{inst:0}} \and
	Bruno Khelifi \inst{\ref{inst:2}} \and
	Ellis Owen \inst{\ref{inst:unknown}} \and
	Brigitta Sipőcz \inst{\ref{inst:6}} \and
	Olga Vorokh \inst{\ref{inst:unknown}} \and
	Julien Lefaucheur \inst{\ref{inst:unknown}} \and
	Fabio Acero \inst{\ref{inst:4}} \and
	Thomas Robitaille \inst{\ref{inst:unknown}} \and
	David Fidalgo \inst{\ref{inst:unknown}} \and
	Jonathan D. Harris \inst{\ref{inst:unknown}} \and
	Lars Mohrmann \inst{\ref{inst:13}} \and
	Cosimo Nigro \inst{\ref{inst:9}} \and
	Dirk Lennarz \inst{\ref{inst:unknown}} \and
	Jalel Eddine Hajlaoui \inst{\ref{inst:unknown}} \and
	Alexis de Almeida Coutinho \inst{\ref{inst:unknown}} \and
	Matthias Wegenmat \inst{\ref{inst:unknown}} \and
	Dimitri Papadopoulos \inst{\ref{inst:unknown}} \and
	Maximilian Nöthe \inst{\ref{inst:unknown}} \and
	Nachiketa Chakraborty \inst{\ref{inst:unknown}} \and
	Michael Droettboom \inst{\ref{inst:14}} \and
	Jason Watson \inst{\ref{inst:5}} \and
	Helen Poon \inst{\ref{inst:unknown}} \and
	Arjun Voruganti \inst{\ref{inst:unknown}} \and
	Vikas Joshi \inst{\ref{inst:unknown}} \and
	Thomas Armstrong \inst{\ref{inst:unknown}} \and
	Erik Tollerud \inst{\ref{inst:16}} \and
	Erik M. Bray \inst{\ref{inst:unknown}} \and
	Domenico Tiziani \inst{\ref{inst:unknown}} \and
	Gabriel Emery \inst{\ref{inst:unknown}} \and
	Hubert Siejkowski \inst{\ref{inst:unknown}} \and
	Kai Brügge \inst{\ref{inst:unknown}} \and
	Luigi Tibaldo \inst{\ref{inst:unknown}} \and
	Arpit Gogia \inst{\ref{inst:unknown}} \and
	Ignacio Minaya \inst{\ref{inst:unknown}} \and
	Marion Spir-Jacob \inst{\ref{inst:unknown}} \and
	Yves Gallant \inst{\ref{inst:unknown}} \and
	Andrew W. Chen \inst{\ref{inst:unknown}} \and
	Roberta Zanin \inst{\ref{inst:unknown}} \and
	Jean-Philippe Lenain \inst{\ref{inst:unknown}} \and
	Larry Bradley \inst{\ref{inst:unknown}} \and
	Kaori Nakashima \inst{\ref{inst:unknown}} \and
	Anne Lemière \inst{\ref{inst:unknown}} \and
	Mathieu de Bony \inst{\ref{inst:unknown}} \and
	Matthew Craig \inst{\ref{inst:unknown}} \and
	Lab Saha \inst{\ref{inst:unknown}} \and
	Zé Vinicius \inst{\ref{inst:unknown}} \and
	Kyle Barbary \inst{\ref{inst:unknown}} \and
	Thomas Vuillaume \inst{\ref{inst:unknown}} \and
	Adam Ginsburg \inst{\ref{inst:unknown}} \and
	Daniel Morcuende \inst{\ref{inst:unknown}} \and
	José Luis Contreras \inst{\ref{inst:unknown}} \and
	Laura Vega Garcia \inst{\ref{inst:unknown}} \and
	Oscar Blanch Bigas \inst{\ref{inst:unknown}} \and
	Víctor Zabalza \inst{\ref{inst:unknown}} \and
	Wolfgang Kerzendorf \inst{\ref{inst:unknown}} \and
	Rolf Buehler \inst{\ref{inst:unknown}} \and
	Sebastian Panny \inst{\ref{inst:unknown}} \and
	Silvia Manconi \inst{\ref{inst:unknown}} \and
	Stefan Klepser \inst{\ref{inst:unknown}} \and
	Peter Deiml \inst{\ref{inst:unknown}} \and
	Johannes Buchner \inst{\ref{inst:unknown}} \and
	Hugo van Kemenade \inst{\ref{inst:unknown}} \and
	Eric O. Lebigot \inst{\ref{inst:unknown}} \and
	Benjamin Alan Weaver \inst{\ref{inst:unknown}} \and
	Debanjan Bose \inst{\ref{inst:unknown}} \and
	Rubén López-Coto \inst{\ref{inst:unknown}} \and
	Sam Carter \inst{\ref{inst:unknown}}
}

\institute{
	APC \label{inst:0} \and
	Astroparticle and Cosmology, CNRS, Université de Paris \label{inst:1} \and
	CNRS \label{inst:2} \and
	Center for Astrophysics | Harvard \& Smithonian \label{inst:3} \and
	DAp/AIM, CEA-Saclay/CNRS \label{inst:4} \and
	DESY \label{inst:5} \and
	DIRAC Institute, UW \label{inst:6} \and
	EMFTEL, UCM \label{inst:7} \and
	HeidelbergCement \label{inst:8} \and
	IFAE \label{inst:9} \and
	INAF/IASF Palermo \label{inst:10} \and
	Instituto de Astrofísica de Andalucía - CSIC \label{inst:11} \and
	MPIK \label{inst:12} \and
	MPIK, Heidelberg \label{inst:13} \and
	Mozilla \label{inst:14} \and
	Pamyra \label{inst:15} \and
	Space Telescope Science Institute \label{inst:16} \and
	unknown \label{inst:unknown}
}

% \abstract{}{}{}{}{} 
% 5 {} token are mandatory

\abstract{
% TODO: the feature are still to detailed and partly copy and pasted form the ICRS abstract...avoid this
% context heading (optional)
% {} leave it empty if necessary  
%{Gammapy context:}
Historically the data as well as analysis software in gamma-ray astronomy is proprietary to the experiments. With the future Cherenkov Telescope Array (CTA), which will be operated as an open gamma-ray observatory with public data, there is a corresponding need for open high-level analysis software. In this article we present the first major version v1.0 of \gammapy, a community-developed open-source Python package for gamma-ray astronomy. 
%
We present its general design and provide an overview of the analysis methods and features it implements. Starting from event lists and a description of the specific instrument response functions (IRF) stored in open FITS based data formats, \gammapy implements . Thereby it handles the dependency of the IRFs with time, energy as well as position on the sky. It offers a variety of background estimation methods for spectral, spatial and spectro-morphological analysis. Counts, background and IRFs data are bundled in datasets and can be serialised, rebinned and stacked.
%
\gammapy supports to model binned data using Poisson maximum likelihood fitting. It comes with built-in spectral, spatial and temporal models as well as support for custom user models, to model e.g. energy dependent morphology of gamma-ray sources. Multiple datasets can be combined in a joint-likelihood approach to either handle time dependent IRFs, different classes of events or combination of data from multiple instruments. Gammapy also implements methods to estimate flux points, including likelihood profiles per energy bin, light curves as well as flux and signficance maps in energy bins.
%
We further describe the general development approach and how \gammapy integrates into ecosystem of other scientific and astronomical Python packages. We also present analysis examples with simulated CTA data and provide results of scientific validation analyses using data of existing instruments such as \hess and \fermi.


% The gamma-ray can be considered as the last frontier 
% In the last two decades the measurement of gamma-ray emission from the universe has 
% evolved from a niche  

% aims heading (mandatory)
%{Gammapy aims}
%Reproducible analyses, share algorithms etc. 


% methods heading (mandatory)
%{Gammapy methods}

% results heading (mandatory)
%{Gammapy results}

% conclusions heading (optional), leave it empty if necessary 
%{Gammapy conclusions}
}

% \date{Received September 15, 1996; accepted March 16, 1997}
\keywords{
	Gamma rays: general -
	Astronomical instrumentation, methods and techniques -
	Methods: data analysis
}

\maketitle

% Main part
\section{Introduction}
\label{sec:introduction}



Very high energy cosmic gamma-ray reaching Earth interact in the atmosphere and
create a large shower of secondary particles that can be observed from the ground.
Ground-based gamma-ray astronomy relies on this phenomenon to detect the
primary gamma-ray photons and measure their direction and energy.
It covers the energy range from fews tens of GeV up to the PeV.
Two main instrument categories exist \citep{2015CRPhy..16..610D}.
Imaging Atmospheric Cherenkov Telescopes (IACT) make images of atmospheric showers
by detecting the Cherenkov radiation emitted by the cascading charged particles and
use these images to reconstruct the properties of the incident particle.

Particle samplers detect directly particles from the tail of the shower when it reaches
the ground. These instruments have very large field-of-view, large duty-cycle but higher
energy threshold and usually have lower signal to noise ratios compared to IACTs.

Ground based gamma-ray astronomy has been historically structured
by experiments run by independent collaborations relying
on their own proprietary data and analysis software developed as part of the
instrument. While this model has been very successful so far, it does not
permit easy combination of data from several instruments and is therefore
a limitation for interoperability of existing facilities.

The Cherenkov Telescope Array (CTA) will be the first instrument to be operated
as an open observatory in the domain. Its data will be shared
Thus there is a need for open analysis software as well.

Moreover, the operation of CTA as an observatory introduces the necessity of
sharing its data publicly. The data-reduction workflow of different IACTs of
the current generation is remarkably similar, resulting in a \textit{high} data
level that can be finally used to derive scientific results (, spectra, sky
maps, light curves). The information in this high data level is independent on
the data reduction, and eventually of the detection technique. This implies,
for example, that data from IACT and WCD can be represented within the same
model. The efforts to prototype the future CTA data model and to convert
current IACT data in a format usable with the newly-available science tools
converged in the so-called \textit{Data Format for Gamma-ray Astronomy}
initiative~\citep{gadf_proc, gadf_universe}, abbreviated in
\texttt{gamma-astro-data-format} (GADF). The latter proposes prototypical
specifications to produce files based on the flexible image transport system
(FITS) format~\citep{fits} encapsulating this high-level information. This is
realized by storing a list of gamma-like events with their measured quantites
(energy, direction, arrival time) and a parametrisation of the response of the
system (see Sec.~\ref{ssec:gammapy-data} and Sec.~\ref{ssec:gammapy-irf} for
more information).

Python has become extremely popular as a scientific programming language
in particular in the field of data sciences. The success is
mostly attributed to the simple and easy to learn syntax, the ability to act as
a "glue" language between different programming languages, the rich eco-system
of packages and the open and supportive community.

 one of the
standard programming  languages for astronomy \footnote{Citation missing} as
well as data sciences in  general \footnote{Citation missing}.

Astronomical data analysis software written in Python existed since 2000. e.g.,
sherpa~\citep{sherpa-2011, sherpa-2009}, or for gamma-ray even PyFACT
~\citep{pyfact}.

The short-term success of Python lead to a prolifaration of packages, until
\astropy~\citep{astropy} was created in 2012. Astropy is and Gammapy is a
Python package for gamma-ray astronomy.

% TODO: describe Context

% TODO: describe goals

TODO: Figure 1: Data -> Gammapy -> Spectra etc with some details

Basic idea: build on Numpy and Astropy, use Python stack

TODO: Figure 2: Gammapy software stack

Here's a list of references I'd like to cite ... to be incorporated into the
main text somewhere:

\begin{itemize}
	\item Gammapy webpage\footnote{\GammapyUrl}
	\item Naima\footnote{\NaimaUrl}~\citep{Naima}
	\item Gammapy use in science publications:~\citep{Owen2015}, SNR shell, HGPS
\end{itemize}

* Gammapy – A Python package for gamma-ray astronomy
* Gammapy – A prototype for the CTA science tools
* Astropy: A community Python package for astronomy
* THE ASTROPY PROJECT: BUILDING AN INCLUSIVE, OPEN-SCIENCE PROJECT AND STATUS
OF THE V2.0 CORE PACKAGE * GammaLib and ctools * Fermipy proceedings * SunPy:
Python for Solar Physics. An implementation for local correlation tracking *

\begin{figure}[t]
	\centering
	\includegraphics[height=0.5\textwidth,
		angle=270]{static/gammapy-big-picture} \caption{ Gammapy is a Python package
		for high-level gamma-ray data analysis. Using event lists, exposures and point
		spread functions as input you can use it to generate science results such as
		images, spectra, light curves or source catalogs. So far it has been used to
		simulate and analyse H.E.S.S., CTA and \textit{Fermi}-LAT data, hopefully it
		will also be applied to e.g., VERITAS, MAGIC or HAWC data in the future. }
	\label{fig:big-picture}

\end{figure}



\section{Gammapy package}
\label{sec:package}

The \gammapy package is structured into multiple sub-packages
which mostly follow the stages in the data reduction workflow.


\subsection{Overview}
\begin{figure*}[t]
\centering
\includegraphics[width=1.\textwidth]{figures/data-flow-gammapy}
\caption{
Gammapy sub-package structure and data analysis workflow.
}
\label{fig:workflow}
\end{figure*}



\subsection{gammapy.data}
The \verb|gammapy.data| sub-package provides access to
DL3 level data and observation handling.


\begin{listing}
\begin{minted}{python}

from gammapy.data import DataStore

data_store = DataStore.from_dir("$GAMMAPY_DATA")
obs_ids = [1, 2, 3]
observations = data_store.get_observations(obs_ids)

\end{minted}
\caption{Using gammapy.data to access DL3 level data with a DataStore}
\label{codeexample:data}
\end{listing}



\subsection{gammapy.makers}
\todo{Regis Terrier}
Data reduction

\begin{listing}
\begin{minted}[numbers=right]{python}

from gammapy.makers import MapDatasetMaker

maker = MapDatasetMaker()
dataset = maker.run(dataset, observation)

\end{minted}
\caption{Using gammapy.data to access DL3 level data}
\label{codeexample:maker}
\end{listing}


\subsection{gammapy.datasets}
\todo{Atreyee Sinha}
DL4 level data


\begin{listing}
\begin{minted}{python}

from astropy.coordinates import SkyCoord
from gammapy.maps import WcsGeom
from gammapy.datasets import MapDataset

skydir = SkyCoord("0d", "0d")
geom = WcsGeom.create(
	skydir=skydir, width="5 deg", binsz="0.2 deg"
)

dataset = MapDataset.create(
	geom=geom, name="my-dataset"
)


\end{minted}
\caption{Using gammapy.data to access DL3 level data with a DataStore}
\label{codeexample:data}
\end{listing}



\subsection{gammapy.modeling}
\todo{Quentin Remy}
Models and fitting

\begin{listing}
\begin{minted}[numbers=right]{python}

from gammapy.modeling.models import (
	SkyModel,
	PowerLawSpectralModel,
	PointSpatialModel
)

pwl = PowerLawSpectralModel()
point = PointSpatialModel()

model = SkyModel(
	spectral_model=pwl,
	spatial_model=point
	name="my-model"
)
\end{minted}
\caption{Using gammapy.data to access DL3 level data}
\label{codeexample:maker}
\end{listing}


\subsection{gammapy.estimators}
\todo{Axel Donath}
Estimators

\subsection{gammapy.visualisation}
Plotters etc.

\subsection{gammapy.analysis}
\todo{Jose Enrique writes this...}
High level analysis API

\subsection{gammapy.astro}
Dark matter models, source population modelling

\subsection{gammapy.data}
\todo{Cosimo Nigro}

\subsection{gammapy.catalog}
Gamma-ray catalog access

\begin{listing}
\begin{minted}{python}

from gammapy.catalog import SOURCE_CATALOGS

\end{minted}
\caption{Using gammapy.data to access DL3 level data with a DataStore}
\label{codeexample:data}
\end{listing}


\subsection{gammapy.maps}
\todo{Laura Olivera-Nieto}
The \verb|gammapy.maps| sub-package provides classes for representing pixelized
data structures with at least two spatial dimensions representing a set of
coordinates or a region on a sphere. In addition it allows to handle and arbitray
number of non-spatial data axes, such as time or energy.

It provides a uniform API for WCS, HEALPix and region based data structures.

\begin{listing}
\begin{minted}{python}

from gammapy.maps import Map
from astropy.coordinates import SkyCoord

skydir = SkyCoord("0d", "5d", frame="galactic")

# Create a WCS Map
m_wcs = Map.create(
	binsz=0.1, map_type='wcs', skydir=skydir, width=10.0
)

# Create a HPX Map
m_hpx = Map.create(
	binsz=0.1, map_type='hpx', skydir=skydir, width=10.0
)

\end{minted}
\caption{Using gammapy.data to access DL3 level data with a DataStore}
\label{codeexample:data}
\end{listing}


\subsection{gammapy.irf}
\todo{Fabio Pintore}
IRF classes

\subsection{gammapy.stats}
\todo{Regis Terrier}
Statistics methods


\subsection{gammapy.utils}
Utility functions...


Outline:
* List typical analysis use cases
* Can use from Python and Jupyter -> show Figure with Jupyter notebook here.
* Gammapy code structure
* How Numpy and Astropy is used


Figures:
* Add a Figure showing dataflow in a typical application
DL3 at the top, spectrum, map, lightcurve, fit results at the bottom.
Mention major classes in between (DataStore, EventList, Map, MapMaker, MapFit, …)
* Probably not: Figure showing sub-packages and how they relate (gammapy.data and gammapy.irf at the base, then gammapy.maps, etc.
* The code example Figure how to make a counts map, to explain how the package works.

\section{Applications}
\label{sec:applications}

Each application example is a notebook in the online material: We could have
one analysis as Python scripts instead of notebook in the online material. At
the start of this section, point to gammapy-paper repo on Github and say that
there’s a Binder where people can try the examples online.

TODO: mention other application examples (joint Crab paper, HESS validation
paper, HGPS, ...) here or in a subsection "other applications" at the end of
this section?

\subsection{Source detection}
\label{ssec:source-detection}

See Figure~\ref{fig:fermi_ts_map}.

Ref:~\citep{Stewart2009}

\subsection{Multi instrument analysis}
\label{ssec:multi-instrument-analysis}
\todo{Cosimo Nigro}

\begin{figure*}[t]
    \sidecaption
	\includegraphics[width=120mm]{figures/multi_instrument_analysis.pdf}
	\caption{A multi-instrument analysis of the Crab Nebula}
	\label{fig:multi_instrument_analysis}
\end{figure*}

\section{Gammapy project}
\label{sec:gammapy-project}

Open development, roadmap, communities, science tool aspect Infrastructure etc.

community driven vs. institutional driven

Validation and benchmarks? Validation as online appendix...

\subsection{Development, testing}
\label{ssec:development-testing}
\todo{Jose Enrique writes this...}
-Github, pytest, CI, PIGs?

\begin{table}
	\import{tables/generated/}{codestats}
	\caption{Coding languages statistics in Gammapy project}
	\label{table:codestats:data}
\end{table}

\begin{figure}[t]
	\centering
	\includegraphics[width=0.5\textwidth]{figures/codestats.pdf}
	\caption{
		Percentage of lines of code in Gammapy project } \label{fig:codestats:lang}
\end{figure}

\subsection{Documentation}
\label{ssec:documentation}

- Notebooks

\subsection{Software distribution and user support}
\label{ssec:software-distribution-and-user-support}
- Pip, conda, versions, gammapy download

\subsection{Community}
\label{ssec:community}
TODO: Figure: Screenshot of Jupyter notebook or docs with notebook, could show
the interactive maps view
\begin{verbatim}
m = Map.read(“diffuse.fits”)
m.plot_interactive()
\end{verbatim}

\subsection{PIGs}
\label{ssec:pigs}

\section{Reproducibility}
\label{sec:reproducibility}
One of the most important goals of the \gammapy project is to support open and
reproducible science results. Thus we decided to write this manuscript
openly and publish the Latex source code along with the associated
Python scripts to create the figures
in \url{https://github.com/gammapy/gammapy-v1.0-paper}.
This GitHub repository also documents the history of the creation
and evolution of the manuscript with time. To simplify the reproducibility
of this manuscript including figures and text, we relied on the tool
"showyourwork"~\citep{Luger2021}. This tool coordinates the building
process and software as well as data dependencies such, that the complete
manuscript can be reproduced with a single \code{make} command, after
downloading the source repository. See \todo{reference?} for detailed
instructions. Almost all figures in this manuscript provide a link
to a Python script, that was used to produce it. This means all
example analyses presented in Sec.\ref{sec:applications} link to
actually working Python source code.


\section{Summary and Outlook}
\label{sec:summary-and-outlook}
%
In this manuscript we presented the first LTS version of \gammapy.
\gammapy is a Python package for \gammaray astronomy, which relies on the
scientific Python ecosystem, including Numpy and Scipy and Astropy as
main dependencies. It also holds the status of an Astropy affiliated
package. It supports high-level analysis of astronomical \gammaray
data from intermediate level data formats, such as the FITS based
\gadf. Starting from lists of \gammaray events and corresponding description
of the instrument response users can reduce and project the data
to WCS, HEALPix and region based data structures. The reduced data is bundled
into datasets, which serve as a basis for Poisson maximum likelihood
modelling of the data. For this purpose \gammapy provides a wide selection
of built-in, spectral, spatial and temporal models as well as unified
fitting interface with connection to multiple optimization backends.

With the v1.0 milestone the \gammapy project enters a new development
phase. Future work will not only include maintenance of the v1.0 release,
but also parallel development of new features, improved API and data
model support. While v1.0 provides all the features required for
standard and advanced astronomical \gammaray data analysis,
we already identified specific improvements to be considered in the
roadmap for a future v2.0 release. This includes the support for
scalable analyses via distributed computing. This will allow
users to scale an analysis from a few observations to multiple
hundreds of observations as expected by deep surveys of the CTA
observatory. In addition the high level interface
of \gammapy is planned to be developed into a fully configurable
API design. This will allow users to define arbitrary complex analysis
scenarios as YAML files and even extend their workflows by user defined
analysis steps via a registry system. Another important topic will
be to improve the support of handling meta data for data structures
and provenance information to track the history of the data reduction
process from the DL3 to the highest DL5/DL6 data levels.

Around the core Python package a large diverse community of
users and contributors has developed. With regular developer meetings,
coding sprints and in-person user tutorials at relevant conferences
and collaboration meetings, the community has constantly grown.
So far \gammapy has seen ~80 contributors from ~10 different countries.
With typically ~10 regular contributors at any given time of the
project, the code base has constantly grown its range of features
and improved its code quality. With \gammapy being selected as the base library
for the future science tools for CTA, we expect the community to grow
even further, providing a stable perspective for further usage,
development and maintenance of the project. Besides the future use
by the CTA community \gammapy has already
been used for analysis of data from the \hess and \magic instruments

While \gammapy was mainly developed for the science community around
IACT instruments, the data model and software design is general
enough to be applied to other \gammaray instruments as well.
The use of \gammapy for the analysis of data from the High Altitude
Water Cherenkov Observatory (HAWC) has been successfully
demonstrated by \todo{Olivera et al.}. This makes \gammapy
a viable choice for the base library for the science tools
of the future Southern Widefield Gamma Ray Observatory
(SWGO) and use with data from LHAASO as well. \gammapy
has the potential to further unify the community
of \gammaray astronomers, by sharing common tools and
a common vision of open and reproducible science for the future.

\begin{acknowledgements}

Mention Christoph here?

We would like to thank the \texttt{Numpy}, \texttt{Scipy}, \texttt{IPython} and
\texttt{Matplotlib} communities for providing their packages which are invaluable
to the development of Astropy. We thank the GitHub team for providing us with
an excellent free development platform. We also are grateful to Read the Docs
(\ReadthedocsUrl), and Travis
(\TravisUrl) for providing free documentation
hosting and testing respectively. Finally, we would like to thank all the
Gammapy users that have provided feedback and submitted bug reports.    
    
TODO: copy over stuff from \url{http://docs.gammapy.org/en/latest/about.html#thanks}.
TODO: add the ANR for Luca (and Atreyee in LUPM?)

\end{acknowledgements}


% Back matter
\bibliographystyle{aa}
\bibliography{bib.bib}

\end{document}
